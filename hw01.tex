%!TEX TS-program = xelatex
\documentclass[12pt, a4paper, oneside]{article}

% Можно вставить разную преамбулу
\input{preamble}

\title{
\begin{center} 
\includegraphics[width=0.99\textwidth]{images/logo.png}
\end{center}

Первая домашка}

\begin{document}

\maketitle

\epigraph{Как говорят у нас на Диком Западе, горячее самогона только кузина, слаще насвая только свобода.}{\textit{Anacondaz, Метафизика}}

Добро пожаловать в первую домашку. Она разбита на три части для вашего удобства. В первой находятся совсем простые задачи. Если вы решите их идеально, вы наберёте $50$ баллов из $100.$ Решая только этот раздел из каждой домашки, вы будете уверенно двигаться к троечке. Если вы претендуете на большее, для вас есть разделы с более сложными задачами. 

Решение работы нужно сдать либо в виде затеханого pdf-файла либо в виде рукописной pdf-ки. Если вы решаете задачи и оформляете их решение на листочке, пишите максимально разборчивым почерком. Если у вас плохой почерк, техайте решения. 

\section*{Задачи на троечку}

Если вы идеально решаете все задачи из этого раздела, вы получаете $50$ баллов из $100$.

\begin{problem}{5}
    В салуне сидят $18$ посетителей. Среди них $12$ пьют виски, $11$ пьют ром, $9$ пьют мескаль. Пятеро пьют виски и ром, четверо виски и мескаль, трое ром и мескаль. Сколько человек пьют все три напитка? Сколько человек пьют только два напитка? Сколько не пьёт ни одного напитка? Изобразите множества на диаграме Эйлера-Вена и подпишите мощности соотвествующих пересечений. 
\end{problem}


\begin{problem}{5}
    Множество исходов $\Omega =\left\{a,b,c\right\}$, $\PP(\left\{a,b\right\})=0,8$,
    $\PP(\{b,c\})=0,7$. Найдите $\PP(\{a\})$, $\PP(\{b\})$, $\PP(\{c\})$.
\end{problem}

\begin{problem}{5}
    Неуловимый Вилли выбирает наугад из четырех тузов разных мастей два. Найдите вероятность того, что они будут разного цвета.
\end{problem}

\begin{problem}{5}
    По пустыне идёт поезд из трёх вагонов. Пусть событие $A_i$ означает, что $i$-ый вагон ограбили. 
    
    \begin{enumerate}
    	\item[а)] Запишите с помощью языка множеств событие $B$ -- ровно два вагона ограбили
    	\item[б)] Запишите с помощью языка множеств событие $B$ -- хотябы один вагон ограбили
    \end{enumerate}
\end{problem}

\begin{problem}{5}
Заключенные в тюрьме ковбои Радомир и Добрыня решили сыграть в игру: Каждый из них подкидывает кость и, если в сумме выпадает $6$, то выигрывает Радомир. А если выпадет $10$, то выигрывает Добрыня. Найдите вероятность того, что победит Добрыня.
\end{problem}


\begin{problem}{5}
Бандит с кликухой Циклоп вышел из своей, как он её назвает, берлоги порезвиться. Он захватил с собою рогатку и 3 камня. Видит — на соседнем ранчо гуляют 20 коров (мало кто знает, но всех их зовут Мартами, потому что - ну а как ещё назвать корову?). 

\begin{enumerate}
    \item[а)] Сколькими способами Циклоп может подбить 3 Март, учитывая то, что для него все коровы одинаковые.
    \item[б)] Как назвать корову? 
\end{enumerate}
\end{problem}

\begin{problem}{10}
Мир пришёл на Великие Равнины. Несколько индейских племён зарыли топор войны и решили раскочегарить трубку мира. 

$20$ вождей случайно рассаживаются в ряд и передают друг-другу трубку. В ряду есть $21$ место. Вождь племени Агвадашинс должен сидеть с краю. Вождей Вабибинэс, Джэки и Маковаян нельзя сажать втроём. 

\begin{enumerate} 
    \item[а)] Вожди садятся в ряд случайно. Какова вероятность того, что рассадка будет правильной и мир окажется заключён? 
    \item[б)] Сколько женщин среди упомянутых четырёх вождей?
\end{enumerate}
\end{problem}

\begin{problem}{10}
    Вероятность застать шерифа в городе зависит от двух факторов: первый — намечается ли в городе стычка, и второй — приедут ли молоденькие девушки из соседнего поселения продавать свой урожай на местную ярмарку. 

    Данная вероятность равна $0.18$, если девушки не приедут и стычка не намечается , $0.9$ — если стычка намечается и девушки приедут. $0.54$ — если намечается только стычка и $0.36$ — если приедут девушки, но стычка не намечается. Стычки происходят с вероятностью $0.4$, а девушки приезжают с вероятностью $0.6$

    Стычки в городе никак не зависят от девушек, несмотря на то, что у каждой из них есть по пушке и ножу.     
    \begin{enumerate} 
        \item[а)] Найдите вероятность того, что в городе что-то происходит (либо стычка либо привезли урожай).
        \item[б)] Найдите вероятность того, что девушки привезли урожай и одновременно с этим началась стычка.
        \item[в)] Найдите вероятность того, что шериф в городе, девушки с урожаем в городе и началась стычка. 
        \item[г)] Найдите вероятность того, что девушки приехали в город на ярмарку, если шериф в городе.
        \item[д)] Найдите вероятность того, что шериф в городе, если известно, что девушки в городе.
    \end{enumerate}
\end{problem}



\section*{Задачи на хор}

Если вы идеально решаете ещё и этот раздел, вы получаете $70$ баллов из $100$.

\begin{problem}{5}
    Бандит «Жадный, но шаловливый» Ричард грабит банк. Ричард забрал себе сейф и развлекается в холе банка с его менеджером.  В сейфе счётное количество купюр. Ровно за час до приезда полиции Бандит выдаёт менеджеру купюры с банковскими номерами 1 и 2 и тут же забирает купюру номер 1 обратно. Ровно за полчаса он выдает менеджеру купюры 3 и 4 и забирает купюру 2 обратно.  Ровно за четверть часа он выдаёт купюры номер 5 и 6 и забирает купюру номер 3. И так далее, ускоряясь выдаёт из сейфа две очередные купюры и забирает у менеджера купюру с наименьшим номером.
    
    \begin{enumerate}
        \item[а)] На сколько изменится количество купюр у менеджера за одну операцию дарения - забирания?
        \item[б)] У кого к приезду полиции окажется купюра номер 1?
        \item[в)] У кого к приезду полиции окажется купюра номер 2015?
        \item[г)] Сколько купюр будет у банка к приезду полиции?
    \end{enumerate}
\end{problem}
% \begin{sol}
% За одну операцию дарения - забирания количество конфет у детишек каждый раз будет увеличиваться на одну.

% Конфета номер 1 к Новому Году окажется у Деда Мороза. Он сразу же отберет её у Вовочки.

% Конфета номер 2015 к Новому Году окажется также у Деда Мороза! Рано или поздно наступит момент, когда он её у Вовочки отберет.

% У Вовочки к Новому Году не будет ни одной конфеты. Дед Мороз всё оставит себе.

\begin{problem}{5}
    Ковбой «Лютый» Джо грабит со своей бандой пассажирский поезд с 9 вагонами. В его команде самые заядлые разбойники: «Мерзкий» Том, «Грязный» Майк, «Меткий» Дик и «Дерзкий» Билли.  
    \begin{enumerate}
        \item[а)] Сколькими способами «Лютый» Джо может рассадить своих ковбоев по вагонам, при условии, что все они должны ехать в различных вагонах.
        \item[б)] Какую кликуху вы бы взяли себе на диком Западе?
    \end{enumerate}
\end{problem}

\newpage

\begin{problem}{5}
    Диаграммы Эйлера-Вена обычно используют, чтобы проиллюстрировать взаимоотношения между элементами множеств. Можно нарисовать корректную диаграмму для двух и трёх множеств. Для четырёх и более множеств корректную диаграмму нарисовать нельзя. Объясните почему. 
\end{problem}

\begin{problem}{5}
    У любой бомбы на диком западе есть шнур длиной $1$ метр. Ковбой Боб случайным образом делает на шнуре два разреза. С какой вероятностью хотя бы один из получившихся кусков будет длиннее $0.5$ метра?
\end{problem}


\section*{Задачи на отл}

Если вы идеально решаете ещё и этот раздел, вы получаете $90$ баллов из $100$.

\begin{problem}{5}
    Сидят в салуне $10$ мексиканцев, $10$ шерифов и $10$ индейцев. К ним вламывается «Малыш» Билли с острым желанием убить каких-нибудь $10$ человек. Какова вероятность, что среди выбранных им $10$ человек не будет как минимум одной категории посетителей салуна?
\end{problem}

\begin{problem}{5}
    Команде шерифов округа Линкольн из $2$-х человек дали задание уничтожить банду грабителей из $10$ человек. Шерифы - люди честные и независимые, поэтому хотят распределить жертв поровну. Так, каждый из шерифов независимо от другого выбирает для себя $5$ грабителей, которых он прихлопнет. Какова вероятность, что их выбор не пересечется?
\end{problem}

\begin{problem}{5}
    Охотнику за головами, Тому, дали большой заказ - пристрелить $20$ участников банды Wild Bunch и $10$ шерифов округа Техас. Чтобы не вызвать переполох в городе, за сегодня он должен убить $7$ человек, но среди них должно быть не менее $1$-ого шерифа. Сколько у Тома способов выполнить свою работу?
\end{problem}


\begin{problem}{5}
    При отсутствии негативных факторов вероятность того, что шериф выживет при отсутствии негативных факторов составляет $0.99.$ При атаке индейцев вероятность выжить равна $0.95$. При появлении в городе бандита вероятность выжить равна $0.9$. Вероятность выжить при одновременной атаке бандита и индейцев равна $0.8$. 

    Бандит приходит в город с вероятностью $0.1,$ индейцы нападают с вероятностью $0.2.$
	
    \begin{enumerate}
        \item[а)]  Какова вероятность гибели шерифа, если считать, что атаки индейцев и приход бандита не зависят друг от друга?
        \item[б)]  Докажите, что вероятность гибели шерифа без предположения о том, что атаки индейцев и приход бандита независимы, заключена в пределах от $0,027$ до $0,033$.
    \end{enumerate} 
\end{problem}

\section*{Задачи на десяточку}

Если вы идеально решаете ещё и этот раздел, вы выбиваете $100$ из $100$. Вы большой молодец. 

\begin{problem}{5}
    На службе у штата Техас находятся три прославленных шерифа: Билл, Иван и Стивен. Каждый из них --- рациональный агент, обладающий списком событий $\mathcal{F}_1$, $\mathcal{F}_2$ и $\mathcal{F}_3$ о правопорядке, который он умеет различать. Все эти списки являются $\sigma$-алгебрами, построенными на одном и том же пространстве элементарных исходов. Приехавший из Европы, молодой и необузданный Француа мечтает стать лучшим шерифом! Три прославленных шерифа соглашаются его обучить. 
    \begin{enumerate}
        \item[а)] Предположим, что Франсуа прошёл последовательное качественное обучение. То есть он сначала узнал всё, что знает Билл, потом всё, что знает Иван, потом всё, что знает Стивен. То есть, Франсуа обладает списком событий $\mathcal{F}_1 \cup \mathcal{F}_2 \cup \mathcal{F}_3$. Является ли этот список событий $\sigma$-алгеброй? 
        
        \item[б)] Предположим, что Франсуа прошёл параллельное обучение и, из-за мешанины в голове, он усвоил только тот материал, который одновременно давали Билл, Иван и Стивен. То есть Франсуа обладает списком событий $\mathcal{F}_1 \cap \mathcal{F}_2 \cap \mathcal{F}_3$. Является ли этот список $\sigma$-алгеброй? 
    \end{enumerate} 
\end{problem}



\begin{problem}{5}
Поезд едет по целочисленным значениям числовой прямой в каком-то направлении с неизвестной скоростью (целое число точек в минуту). Вам неизвестны позиция и скорость поезда. 

Вы хотите ограбить поезд. В вашей шайке есть неограниченное количество бандитов. Вы можете каждую минуту посылать в какую-либо целочесленную точку числовой прямой бандита. Если поезд в данный момент находится в этой точке, бандит его ограбит. Время не ограничено. Придумайте стретегию, которая поможет гарантированно ограбить поезд. 
\end{problem}
\end{document}
